\documentclass[a4paper]{article}
 \usepackage[a4paper,width=150mm,top=25mm,bottom=25mm,bindingoffset=6mm]{geometry} 
%\usepackage{simplemargins}
%\usepackage[square]{natbib}
\usepackage{cite}
\usepackage{amsmath}
\usepackage{amsfonts}
\usepackage{amssymb}
\usepackage{apacite}
\usepackage{graphicx}


\begin{document}
\pagenumbering{gobble}


\Large
 \begin{center}
   \textit{Heute intervenieren, Morgen beraten, Übermorgen vorbeugen. \\ Der Drei-Wege-Plan.} \\ Vorschlag für ein betriebliches Suchtpräventionsprogramm an der Universitätsmedizin Mannheim \\
\hspace{10pt}

% Author names and affiliations
\large
Simon Mangold$^1$, Finn Überrück-Fries$^1$, Judith Jessberger$^1$ \\

\hspace{10pt}

\small 
$^1$ Medizinische Fakultät Heidelberg, MIPH - Mannheimer Institut für Public Health\\
s.mangold@stud.uni-heidelberg.de, finn@ueberfries.de, judith.jessberger@gmx.de\\

\end{center}


\normalsize
% Global -> WHO, ILO, Inzidenz, Epidemiologie, VWL-Schaden
\subsubsection*{Hintergrund}% \mbox{} \\ \hspace{2mm}
Von 2009 bis 2019 stieg die absolute Prävalenz von Suchterkrankungen (ICD F10-F19) bei erwerbstätigen AOK-Versicherten um 35,9\% an \shortcite{baduraFehlzeitenReport2020Gerechtigkeit2020}. 2018 entfielen je 1000 Versicherten im Gesundheitssektor 164 Arbeitsunfähigkeitstage (AU) auf Alkoholmissbrauch \shortcite{kniepsPsychischeGesundheitUnd2019}. Meist mit Nikotinabusus und entsprechenden Komorbiditäten vergesellschaftet, entspricht dies mehr als 80\% der suchtbedingten AU Tage \shortcite{badura2013}.  
% Lokal -> wichtige Paper (max 2-3), Interventionen
%In einer \cite{croissant200}. a

% Ökonomie, Kosten Nutzen

%\cite{goecke2018}
%\cite{loeber2008}
%\cite{weber2016}
%\shortcite{wienemann2011

\subsubsection*{Material \& Methoden}


In die Literaturrecherche einbezogen wurden wissenschaftliche Publikationen aus Pubmed, Jahresberichte der AOK und BKK, die Standards zur betrieblichen Präventionsprogramme der Deutschen Hauptstelle für Suchtfragen und die Berichte der Bundesdrogenbeauftragen. Darauf basierend wurde ein betriebliches Suchtpräventionsprogramm erstellt. Die Kosten-Nutzen Analyse konnte mittels den Ergebnissen einer Fall-Kontroll Studie  \shortcite{ennenbach2009} und der aktuellen Prävalenz der Beschäftigten im Gesundheitsberuf evaluiert werden.

\subsubsection*{Ergebnisse}


Zunächst erfolgt in der UMM eine anonymen Erhebung \shortcite{ennenbach2009} des Ist-Zustand. Eine spezifische Erweiterung des Fragebogens soll der Mehrbelastung der Intensivmedizin unter SARS-Cov2 gerecht werden \shortcite{AnestheticDrug2017}. Auf Grundlage dieser Ergebnisse wird anschließend ein dreistufiges Interventionsprogramm gezielt erstellt. Zur langfristigen Durchführung sollen (nach derzeitigen Empfehlungen \shortcite{wienemann2011}) zwei Suchtbeauftragte hauptamtlich angestellt werden. \textit{Heute intervenieren}: Zur Verhältnisprävention wird ein Klinikleitbild erstellt (Alkoholverbot, kostenloses Mineralwasser, Reduktion und Verlegung der Raucherbereiche an den Neckar). Auf individueller Ebene soll eine anonyme Suchtambulanz Hilfe bieten. \textit{Morgen beraten}: Eine Schulung der Führungskräfte in einer dreitägigen Seminarreihe schafft zusätzliche Sensibilisierung, Früherkennung und Sicherheit im Umgang mit Sucht und Suchterkrankten . \textit{Übermorgen vorbeugen}: Langfristig soll die Unternehmenskultur durch regelmäßige Seminare entlang eines Leitbilds des \textit{servant leader} und einer \textit{leader humility} umgestaltet werden. Durch eine Erweiterung des Suchtbegriffs (z.B. Arbeitssucht) wird modernen Entwicklungen (Burn-Out) Vorbeugung geleistet. Es wäre nach ca. 3 Jahren zu erwarten, dass die Zahl der AU-Tage um ca. 20\% sowie der Alkoholkonsum signifikant gesenkt werden können und sich die Zufriedenheit der Mitarbeiter signifikant erhöht \shortcite{ennenbach2009}.

\subsubsection*{Diskussion}


Zu interventionellen BSPs ist die Datenlage knapp und wenig aktuell, weshalb auf die gesellschaftliche Dynamik nur unzureichend reagiert werden kann. So stellen Ennebach et al. (2009) fest, dass die Interventionen interaktiver sein sollten (mehr Rollenspiel und Feedback). Interventionen als Vermittler von innovativen, ganzheitlichen Ideen zur Umsetzung eines Präventionsprogramms können somit ein Nadelöhr bilden. Konstatierend schaffen eine ethische und gesundheitsförderliche Führungskultur langfristig eine Nutzenmaximierung durch Wirtschaftlichkeit (Fehlzeit, Leistung, Arbeitsklima) sowie Gesundheit (Zufriedenheit, physische und psychosoziale Ressourcen) \shortcite{jimenezEnhancingResourcesWorkplace2017}, \shortcite{baduraFehlzeitenReport2020Gerechtigkeit2020}.
\newpage
\renewcommand{\refname}{Literaturverzeichnis}
\bibliography{Literatur/MeineBibliothek.bib}
\bibliographystyle{apacite}

\end{document}
